\documentclass{article}

\usepackage{amsmath}
\usepackage{graphicx}
\usepackage{hyperref}
\usepackage{cite}

\title{Knowledge Graph Attention Network Review}
\author{Babiński Michał \\
        Dymanowski Krzysztof \\
        Wesołowski Jędrzej \\
        Gdańsk University of Technology \\
        \texttt{TODO}}

\begin{document}

\maketitle

\begin{abstract}
Knowledge graphs are cool man.
\end{abstract}

\section{Introduction}
What are knowledge graphs? Why are they important? What is the problem the authors are trying to solve by using KG in an attention network for a recommender algorithm?

\section{Task 1}
W pierwszym etapie projektu studenci wybierają artykuł naukowy
z zadanej przez prowadzącą puli dostępnych artykułów i analizują go.
W pierwszej kolejności studenci powinni skupić się na poruszanym w artykule
problemie naukowym oraz na opisie istniejących sposobów podejścia do
danego problemu (o ile istnieją). Jeżeli takie rozwiązania istnieją, należy
dowiedzieć się, jakie są ich zalety oraz wady, które zachęciły autorów
wybranego artykułu do podjęcia danej tematyki.
Kolejnym krokiem w Etapie I jest zrozumienie zaproponowanego
w artykule algorytmu lub metody. Studenci powinni szczegółowo przeanalizować
przebieg algorytmu, wszystkie parametry stałe i zmienne oraz format
przyjmowanych i zwracanych danych oraz etap procesu rekomendacji, w którym
algorytm/metoda zostały dodane/zmodyfikowane (w zależności od wybranego
przez studenta artykułu).
Na zakończenie pierwszego etapu studenci powinni dokonać refleksji nad
poznanym algorytmem oraz problemem, który miał rozwiązywać i odpowiedzieć
na kilka pytań:
\begin{itemize}
    \item Jakie są wady i zalety proponowanej w artykule metody?
    \item Czym różni się od wcześniejszych rozwiązań?
    \item Czy jest uniwersalna? Jeśli nie, to jakie są jej ograniczenia?
    \item Czy rzeczywiście rozwiązuje opisywany problem? Jeśli nie, to dlaczego?
    \item Czy widzisz sposób na ulepszenie/modyfikację proponowanego w artykule algorytmu? Jeśli tak, to na czym ta modyfikacja polega?
\end{itemize}

\subsection{Sprawozdanie do Etapu I}
Sprawozdanie z Etapu I powinno zawierać podstawowe informacje
o autorach, tj. imiona, nazwiska i numery indeksów wszystkich członków
zespołu projektowego oraz informacje o analizowanym artykule, tj. nazwiska
autorów i tytuł. 
Poniżej podano wymagane elementy merytoryczne sprawozdania.
\begin{enumerate}
    \item Motywacje
    \begin{enumerate}
        \item Poruszany w artykule problem z opisem, dlaczego jest ważny dla dziedziny.
        \item Istniejące rozwiązania problemu wraz z ich zaletami i wadami.
        \item Informacja, dlaczego autorzy analizowanego artykułu zajęli się tą tematyką.
        \item Dlaczego autorzy uważają, że ich metoda rozwiąże lub pomoże rozwiązać w przyszłości dany problem.
    \end{enumerate}
    \item Algorytm
    \begin{enumerate}
        \item Szczegółowy opis algorytmu, najlepiej dodatkowo zilustrowany pseudokodem lub schematem blokowym.
        \item Identyfikacja wszystkich parametrów algorytmu:
        \begin{enumerate}
            \item stałe wraz z ich wartościami
            \item zmienne wraz z możliwym zakresem (i co się zmienia dla różnych wartości)
        \end{enumerate}
        \item Identyfikacja formatu danych:
        \begin{enumerate}
            \item Wymagany format danych wejściowych.
            \item Format zwracanych danych.
        \end{enumerate}
        \item Analiza procesu rekomendacji – w którym etapie procesu rekomendacji proponowany algorytm jest wykorzystywany.
    \end{enumerate}
    \item Wnioski
    \begin{enumerate}
        \item Jakie są wady i zalety proponowanej w artykule metody?
        \item Czym różni się od wcześniejszych rozwiązań?
        \item Czy jest uniwersalna? Jeśli nie, to jakie są jej ograniczenia?
        \item Czy rzeczywiście rozwiązuje opisywany problem? Jeśli nie, to dlaczego?
        \item Czy widzisz sposób na ulepszenie/modyfikację proponowanego w artykule algorytmu? Jeśli tak, to na czym ta modyfikacja polega?
    \end{enumerate}
\end{enumerate}

\section{Task 2}
W drugim etapie projektu studenci analizują kod dołączony do wybranego
w pierwszym etapie artykułu naukowego. W pierwszej kolejności studenci
powinni skupić się na sposobie zaimplementowania proponowanej przez
autorów artykułu metody, w tym na poprawność implementacji oraz ciekawe
rozwiązania techniczne.
W kolejnym kroku studenci powinni przeanalizować sposób
przeprowadzenia eksperymentu. Jaki zbiór danych został użyty? W jaki sposób
dokonano podziału danych? Z jakich metod i miar skorzystano? Czy protokół
oceny algorytmu odpowiada problemowi, który autorzy artykułu chcieli
rozwiązać?
W ostatniej części etapu studenci powinni powtórzyć eksperyment opisany
w wybranym artykule z wykorzystaniem analizowanego we wcześniejszych
krokach kodu. Co należy zrobić, aby uruchomić kod? Czy wystarczy jedno
uruchomienie, czy np. istnieje plik konfiguracyjny, który należy modyfikować,
aby otrzymać wszystkie raportowane w artykule wyniki? Czy wyniki zgadzają się
z tym, które zostały zaraportowane w artykule przez autorów? Jeżeli nie, to
dlaczego tak jest? Jakie mogą być tego konsekwencje?

\subsection{Prezentacja do Etapu II}
Prezentacja z Etapu II powinna zawierać podstawowe informacje
o autorach, tj. imiona, nazwiska i numery indeksów wszystkich członków
zespołu projektowego oraz informacje o analizowanym artykule, tj. nazwiska
autorów i tytuł. 
Poniżej podano elementy merytoryczne prezentacji. (Jeżeli czegoś z poniższych
nie można zastosować dla konkretnego tematu, to powinno zostać pominięte – 
ewentualnie umieszczone na slajdzie, ale nie omówione podczas prezentacji.)
\begin{enumerate}
    \item Omówienie algorytmu, czyli streszczenie etapu 1 celem zrozumienia
    dalszej części prezentacji przez kolegów (tylko i aż tyle!).
    \item Analiza kodu
    \begin{enumerate}
        \item Poprawność implementacji (błędy metody i merytoryczne, tj.
        niezgodność z opisem metody).
        \item Ciekawe rozwiązania techniczne, np. optymalizacja obliczeń itp.
    \end{enumerate}
    \item Opis użytego w eksperymencie zbioru/zbiorów danych
    \begin{enumerate}
        \item Jaki był cel zbierania zbioru?
        \item Kto jest twórcą/właścicielem zbioru?
        \item Jakie dane zawiera?
        \item Co one oznaczają?
        \item Jakie są wartości dla podstawowych charakterystyk zbioru danych?
        \begin{enumerate}
            \item Ilość ocen, użytkowników i produktów.
            \item Średnie ilości ocen na użytkownika/produkt.
            \item Gęstość macierzy ocen.
            \item Dodatkowe informacje zawarte w zbiorze charakterystyczne dla
            rozwiązywanego problemu/proponowanej metody.
        \end{enumerate}
    \end{enumerate}
    \item Analiza eksperymentu
    \begin{enumerate}
        \item Z jakich metod ewaluacji skorzystano?
        \item Z jakich miar oceny systemów rekomendacyjnych skorzystano? Dlaczego?
        \item Czy wykorzystano odpowiedni zbiór danych do rozwiązywanego
        problemu?
        \item W jaki sposób dokonano podziału danych? Czy przeprowadzano na
        nich jakieś dodatkowe transformacje?
        \item Czy protokół oceny algorytmu odpowiada problemowi, który autorzy
        artykułu chcieli rozwiązać?
    \end{enumerate}
    \item Wykonanie kodu
    \begin{enumerate}
        \item W jaki sposób uruchamia się program?
        \item Czy istnieją jakieś pliki konfiguracyjne? Jakie? Do czego służą?
        \item Jak długo trzeba czekać na wyniki? (nie wymaga się podawania
        dokładnych danych).
    \end{enumerate}
    \item Wyniki i wnioski
    \begin{enumerate}
        \item Wyniki własnego wykonania w zestawieniu z wynikami z artykułu.
        \item Czy wyniki się różnią?
        \item Jaka może być tego przyczyna?
        \item Jakie mogą być tego konsekwencje?
        \item Czy uważasz, że algorytm został poprawnie zwalidowany? Dlaczego?
        \item Czy widzisz sposób na ulepszenie/modyfikację tego eksperymentu?
    \end{enumerate}
\end{enumerate}

\section{Task 3}
W trzecim etapie projektu studenci, korzystając z wiedzy zdobytej 
w poprzednich dwóch etapach projektu, proponują własną modyfikację algorytmu 
i/lub eksperymentu przeprowadzonego przez autorów oryginalnego artykułu i 
planują eksperyment mający na celu weryfikację zaproponowanej modyfikacji.
Na początku studenci powinni przemyśleć możliwe modyfikacje, które 
zidentyfikowali w poprzednich sprawozdaniach. Można też rozszerzyć pulę 
o nowe propozycje. Studenci powinni opisać wszystkie zidentyfikowane przez 
siebie możliwe modyfikacje, zarówno samego algorytmu, jak i eksperymentu (nie 
obliguje to do implementacji wszystkich rozszerzeń w kolejnym etapie).
Podczas opisu każdej z możliwych modyfikacji algorytmu i/lub 
eksperymentu, studenci powinni rozważyć następujące kwestie:
\begin{itemize}
    \item Na czym polega dana modyfikacja?
    \item W jaki sposób zmieni ona sposób wykonania algorytmu/eksperymentu?
    \item W którym miejscu w kodzie będą wprowadzone zmiany?
    \item Jakie są motywacje wprowadzenia danej modyfikacji?
    \item Czy pozwoli ona ulepszyć algorytm/eksperyment? W jaki sposób?
    \item W przypadku modyfikacji eksperymentu powinno się rozważyć również pytania:
    \begin{itemize}
        \item Jaką nową wiedzę o metodzie możemy zyskać dzięki danej modyfikacji eksperymentu?
        \item Do czego w praktyce może się ta wiedza przydać?
    \end{itemize}
\end{itemize}
W ostatniej części etapu trzeciego studenci wybierają dwie modyfikacje i 
planują dla nich eksperymenty. Należy również uzasadnić wybór modyfikacji do 
implementacji.
Plan eksperymentu powinien zawierać informacje o tym, co będzie badane 
i w jaki sposób. Studenci powinni zidentyfikować wszystkie zmienne zależne i
niezależne, w tym również hiperparametry wymagające wcześniejszego 
strojenia. Każdy krok eksperymentu powinien być dokładnie zaplanowany i 
opisany. Ważny jest również wybór i opis zbiorów danych do eksperymentu.

\subsection{Prezentacja do Etapu III}
Prezentacja z Etapu III powinna zawierać podstawowe informacje 
o autorach, tj. imiona, nazwiska i numery indeksów wszystkich członków zespołu 
projektowego oraz informacje o analizowanym artykule, tj. nazwiska autorów i 
tytuł. 
Poniżej podano wymagane elementy merytoryczne sprawozdania.
Prezentacja powinna trwać 10-12 minut.
\begin{enumerate}
    \item Możliwe modyfikacje
    \begin{itemize}
        \item Dla każdej zidentyfikowanej możliwej modyfikacji, studenci powinni rozważyć następujące elementy:
        \begin{enumerate}
            \item Istota modyfikacji
            \begin{itemize}
                \item Typ modyfikacji (algorytmu czy eksperymentu).
                \item Na czym polega dana modyfikacja?
                \item W jaki sposób zmieni ona sposób wykonania algorytmu/eksperymentu?
                \item W którym miejscu w kodzie będą wprowadzone zmiany?
            \end{itemize}
            \item Motywacje do wprowadzenia danej modyfikacji
            \begin{itemize}
                \item Jakie są motywacje wprowadzenia danej modyfikacji?
                \item Czy pozwoli ona ulepszyć algorytm/eksperyment? W jaki sposób?
                \item Opcjonalne (tylko w przypadku modyfikacji eksperymentu): Jaką nową wiedzę o metodzie możemy zyskać dzięki danej modyfikacji eksperymentu? Do czego w praktyce może się ta wiedza przydać?
            \end{itemize}
        \end{enumerate}
    \end{itemize}
    \item Plan eksperymentu
    \begin{enumerate}
        \item Wybór modyfikacji
        \begin{itemize}
            \item Która z powyższych modyfikacji została wybrana do implementacji? Dlaczego akurat ta?
        \end{itemize}
        \item Opis eksperymentu
        \begin{itemize}
            \item Dokładny opis planowanego eksperymentu.
            \item Jakie zbiory danych zostaną wykorzystane?
            \item Jakie kroki zostaną przeprowadzone?
            \item Co będziemy badać?
            \item Jakie będą zmienne zależne a jakie niezależne w projektowanym eksperymencie?
            \item Czy będą wykorzystywane miary jakości systemów rekomendacyjnych? Jeśli tak, to jakie?
            \item Czy wymagane jest wcześniejsze strojenie hiperparametrów modelu? Jeżeli tak, to których i w jaki sposób będzie wykonane?
        \end{itemize}
    \end{enumerate}
\end{enumerate}

\section{Task 4}
W czwartym etapie projektu studenci, korzystając z planu eksperymentu
zaprojektowanego w trzecim etapie, przeprowadzają eksperyment naukowy,
analizują jego wyniki i formułują wnioski.
W pierwszej części czwartego etapu studenci powinni wykonać dwa
eksperymenty zgodnie z planem opisanym w sprawozdaniu do etapu trzeciego.
Istotna jest tutaj zgodność z opisanym planem. Jeżeli wystąpią jakiekolwiek
trudności z realizacją zaplanowanych w eksperymencie czynności, studenci
powinni opisać te problemy wraz z rozwiązaniem, które zostało zastosowane.
Wszelkie modyfikacje planu również powinny zostać skrupulatnie opisane.
Po wykonaniu każdego z eksperymentów, studenci powinni zebrać
otrzymane wyniki w wygodnej do analizy formie, np. wykres, tabela. Dobrze by
było, gdyby zostały przedstawione w takiej formie, która umożliwia porównanie
otrzymanych wyników z tymi z oryginalnego artykułu (w uzasadnionych
przypadkach nie jest to konieczne). Otrzymane wyniki powinny zostać
przeanalizowane samodzielnie, jak również porównane z tymi z oryginalnego
artykułu (o ile to możliwe). Na podstawie planu modyfikacji i wyników oraz
oryginalnego artykułu, studenci powinni wyciągnąć wnioski z przeprowadzonych
badań (dla każdego eksperymentu osobno).
W ostatniej części etapu czwartego studenci podsumowują pracę ze
wszystkich dotychczasowych etapów i wyciągają wnioski końcowe. Kluczowe
jest zidentyfikowanie, czy podczas przeprowadzonych eksperymentów powstała
nowa wiedza o badanym algorytmie (i opisanie jej, jeśli tak). Możliwe jest tu
uwzględnienie wszelkich uwag dotyczących oryginalnej metody, procedury
ewaluacji, zaimplementowanych modyfikacjach i otrzymanych wyników.

\subsection{Prezentacja do Etapu IV}
Prezentacja z Etapu IV powinna zawierać podstawowe informacje
o autorach, tj. imiona, nazwiska i numery indeksów wszystkich członków
zespołu projektowego oraz informacje o analizowanym artykule, tj. nazwiska
autorów i tytuł. 
Poniżej podano ramowe elementy merytoryczne sprawozdania.
\begin{enumerate}
    \item Analizowany algorytm
    \begin{itemize}
        \item Krótkie przypomnienie, co robi analizowany algorytm.
    \end{itemize}
    \item Eksperymenty
    \begin{enumerate}
        \item Przebieg eksperymentu
        \begin{itemize}
            \item Dokładny opis przeprowadzonego eksperymentu.
            \item Czy udało się przeprowadzić modyfikację zgodnie z założonym planem? Jeśli nie, to co się zmieniło?
            \item Czy wystąpiły jakieś niespodziewane problemy? Jeśli tak, to jakie?
        \end{itemize}
        \item Wyniki
        \begin{itemize}
            \item Dokładny opis otrzymanych wyników z eksperymentu wraz z tabelami i/lub wykresami.
        \end{itemize}
        \item Porównanie wyników i wnioski
        \begin{itemize}
            \item Czy wyniki różnią się od wyników otrzymanych przez autorów oryginalnego artykułu? Jeśli tak, to czy są lepsze czy gorsze? Jak myślisz, dlaczego tak jest?
            \item Jeśli modyfikacją było nowe zastosowanie metody, to czy po przeprowadzonym eksperymencie możemy uznać, że metoda sprawdzi się w tym zastosowaniu? Dlaczego? Itd.
        \end{itemize}
    \end{enumerate}
    \item Wnioski końcowe
    \begin{itemize}
        \item Co możesz powiedzieć o oryginalnym algorytmie i własnych modyfikacjach po przeprowadzeniu eksperymentów?
        \item Czy możesz wysnuć jakieś wnioski z wszystkich eksperymentów (z artykułu i własnych)?
        \item Czy powstała nowa wiedza podczas realizacji projektu? Jeśli tak, to jaka? Itp.
    \end{itemize}
\end{enumerate}

\section{Task 5}
\subsection{Przebieg Etapu V}
W piątym etapie projektu studenci przygotowują artykuł naukowy zgodnie
z załączonym szablonem.
Artykuł powinien zawierać wszystkie kluczowe informacje dotyczące
wszystkich wcześniejszych etapów projektu, ze szczególnym naciskiem na etap
czwarty (własny eksperyment i wnioski).

\subsection{Sprawozdanie do Etapu V}
Sprawozdanie z Etapu V będzie stanowił artykuł naukowy, który powinien
zawierać podstawowe informacje o autorach, tj. imiona i nazwiska wszystkich
członków zespołu projektowego oraz informacje o analizowanym artykule, tj.
nazwiska autorów i tytuł. Ponadto, artykuł powinien zawierać następujące
elementy:
\begin{enumerate}
    \item Wprowadzenie do problematyki.
    \item Przegląd stanu wiedzy/istniejących modyfikacji algorytmu
    \item Analizowany algorytm
    \item Przeprowadzone eksperymenty wraz z wynikami
    \item Wnioski końcowe
\end{enumerate}
Dokładny dobór treści w ramach punktów jest pozostawiony studentom
i podlega ocenie.
Artykuł powinien być napisany w języku angielskim z wykorzystaniem
załączonego szablonu LaTeX. Długość artykułu powinna wynosić 10-14 stron
(łącznie z bibliografią).

\subsection{Forma i sposób zaliczenia Etapu V}
Za etap można otrzymać maksymalnie 8 punktów. Do zaliczenia zadania
wymagane jest otrzymanie minimum 4 punkty. Oceniany będzie dobór i sposób
zaprezentowania treści. Termin wgrania artykułów upływa 21.01.2024 o godz.
23:59.

\section{Conclusions}
This section acknowledges the contributions of individuals and organizations that supported the research.

\bibliographystyle{plain}
\bibliography{references}

\end{document}