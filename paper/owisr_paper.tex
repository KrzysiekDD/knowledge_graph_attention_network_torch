\documentclass[a4paper]{LTJournalArticle}

\addbibresource{sample.bib} % BibLaTeX bibliography file

\runninghead{Shortened Running Article Title} % A shortened article title to appear in the running head, leave this command empty for no running head

\footertext{\textit{Journal of Biological Sampling} (2024) 12:533-684} % Text to appear in the footer, leave this command empty for no footer text

\setcounter{page}{1} % The page number of the first page, set this to a higher number if the article is to be part of an issue or larger work

%----------------------------------------------------------------------------------------
%	TITLE SECTION
%----------------------------------------------------------------------------------------

\title{Knowledge Graph Attention Network Review}
\author{Babiński Michał \\
        Dymanowski Krzysztof \\
        Wesołowski Jędrzej \\
        Gdańsk University of Technology \\
        \texttt{TODO}}
% Authors are listed in a comma-separated list with superscript numbers indicating affiliations
% \thanks{} is used for any text that should be placed in a footnote on the first page, such as the corresponding author's email, journal acceptance dates, a copyright/license notice, keywords, etc
% \author{%
% 	John Smith\textsuperscript{1,2}, Robert Smith\textsuperscript{3} and Jane Smith\textsuperscript{1}\thanks{Corresponding author: \href{mailto:jane@smith.com}{jane@smith.com}\\ \textbf{Received:} October 20, 2023, \textbf{Published:} December 14, 2023}
% }

% Affiliations are output in the \date{} command
\date{\footnotesize\textsuperscript{\textbf{1}}School of Chemistry, The University of Michigan\\ \textsuperscript{\textbf{2}}Physics Department, The University of Wisconsin\\ \textsuperscript{\textbf{3}}Biological Sciences Department, The University of Minnesota}

% Full-width abstract
\renewcommand{\maketitlehookd}{%
	\begin{abstract}
		\noindent Lorem ipsum dolor sit amet, consectetur adipiscing elit. Praesent porttitor arcu luctus, imperdiet urna iaculis, mattis eros. Pellentesque iaculis odio vel nisl ullamcorper, nec faucibus ipsum molestie. Sed dictum nisl non aliquet porttitor. Etiam vulputate arcu dignissim, finibus sem et, viverra nisl. Aenean luctus congue massa, ut laoreet metus ornare in. Nunc fermentum nisi imperdiet lectus tincidunt vestibulum at ac elit. Nulla mattis nisl eu malesuada suscipit. Aliquam arcu turpis, ultrices sed luctus ac, vehicula id metus. Morbi eu feugiat velit, et tempus augue. Proin ac mattis tortor. Donec tincidunt, ante rhoncus luctus semper, arcu lorem lobortis justo, nec convallis ante quam quis lectus. Aenean tincidunt sodales massa, et hendrerit tellus mattis ac. Sed non pretium nibh. Donec cursus maximus luctus. Vivamus lobortis eros et massa porta porttitor.
	\end{abstract}
}

%----------------------------------------------------------------------------------------

\begin{document}
	
	\maketitle % Output the title section
	
	%----------------------------------------------------------------------------------------
	%	ARTICLE CONTENTS
	%----------------------------------------------------------------------------------------
	
	\section{Introduction}
	
	Lorem ipsum dolor sit amet, consectetur adipiscing elit. Praesent porttitor arcu luctus, imperdiet urna iaculis, mattis eros. Pellentesque iaculis odio vel nisl ullamcorper, nec faucibus ipsum molestie. Sed dictum nisl non aliquet porttitor. Etiam vulputate arcu dignissim, finibus sem et, viverra nisl. Aenean luctus congue massa, ut laoreet metus ornare in. Nunc fermentum nisi imperdiet lectus tincidunt vestibulum at ac elit. Nulla mattis nisl eu malesuada suscipit.
	
	Aliquam arcu turpis, ultrices sed luctus ac, vehicula id metus. Morbi eu feugiat velit, et tempus augue. Proin ac mattis tortor. Donec tincidunt, ante rhoncus luctus semper, arcu lorem lobortis justo, nec convallis ante quam quis lectus. Aenean tincidunt sodales massa, et hendrerit tellus mattis ac. Sed non pretium nibh. Donec cursus maximus luctus. Vivamus lobortis eros et massa porta porttitor. Donec laoreet nisl vel risus lacinia elementum non nec lacus. Nullam luctus, nulla volutpat ultricies ultrices, quam massa placerat augue, ut fringilla urna lectus nec nibh. Vestibulum efficitur condimentum orci a semper. Pellentesque ut metus pretium lacus maximus semper.
	
	Fusce varius orci ac magna dapibus porttitor. In tempor leo a neque bibendum sollicitudin. Nulla pretium fermentum nisi, eget sodales magna facilisis eu. Praesent aliquet nulla ut bibendum lacinia. Donec vel mauris vulputate, commodo ligula ut, egestas orci. Suspendisse commodo odio sed hendrerit lobortis. Donec finibus eros erat, vel ornare enim mattis et. Donec finibus dolor quis dolor tempus consequat. Mauris fringilla dui id libero egestas, ut mattis neque ornare. Ut condimentum urna pharetra ipsum consequat, eu interdum elit cursus. Vivamus scelerisque tortor et nunc ultricies, id tincidunt libero pharetra. Aliquam eu imperdiet leo. Morbi a massa volutpat velit condimentum convallis et facilisis dolor.
	
	\begin{equation}
		\cos^3 \theta =\frac{1}{4}\cos\theta+\frac{3}{4}\cos 3\theta
		\label{eq:example}
	\end{equation}
	
	Automatically referencing an equation number using its label: Equation \ref{eq:example}.
	
	In hac habitasse platea dictumst. Curabitur mattis elit sit amet justo luctus vestibulum. In hac habitasse platea dictumst. Pellentesque lobortis justo enim, a condimentum massa tempor eu. Ut quis nulla a quam pretium eleifend nec eu nisl. Nam cursus porttitor eros, sed luctus ligula convallis quis. Nam convallis, ligula in auctor euismod, ligula mauris fringilla tellus, et egestas mauris odio eget diam. Praesent sodales in ipsum eu dictum. Aenean vel enim ipsum. Fusce ut felis at eros sagittis bibendum mollis lobortis libero.
	
	Maecenas consectetur metus at tellus finibus condimentum. Proin arcu lectus, ultrices non tincidunt et, tincidunt ut quam. Integer luctus posuere est, non maximus ante dignissim quis. Nunc a cursus erat. Curabitur suscipit nibh in tincidunt sagittis. Nam malesuada vestibulum quam id gravida. Proin ut dapibus velit. Vestibulum eget quam quis ipsum semper convallis. Duis consectetur nibh ac diam dignissim, id condimentum enim dictum. Nam aliquet ligula eu magna pellentesque, nec sagittis leo lobortis. Aenean tincidunt dignissim egestas.
	
	%------------------------------------------------
	
	\section{Methodologies}
	
	\subsection{Sample Sites \& Processing}
	
	This line shows how to use a footnote to further explain or cite text\footnote{Example footnote text.}.
	
	This is a bullet point list:
	
	\begin{itemize}
		\item Arcu eros accumsan lorem, at posuere mi diam sit amet tortor
		\item Fusce fermentum, mi sit amet euismod rutrum
		\item Sem lorem molestie diam, iaculis aliquet sapien tortor non nisi
		\item Pellentesque bibendum pretium aliquet
	\end{itemize}
	
	Mauris interdum porttitor fringilla. Proin tincidunt sodales leo at ornare. Donec tempus magna non mauris gravida luctus. Cras vitae arcu vitae mauris eleifend scelerisque. Nam sem sapien, vulputate nec felis eu, blandit convallis risus. Pellentesque sollicitudin venenatis tincidunt. In et ipsum libero. Nullam tempor ligula a massa convallis pellentesque.
	
	This is a numbered list:
	
	\begin{enumerate}
		\item Donec dolor arcu, rutrum id molestie in, viverra sed diam
		\item Curabitur feugiat
		\item Turpis sed auctor facilisis
	\end{enumerate}
	
	\subsection{Species Identification}
	
	Proin lobortis efficitur dictum. Pellentesque vitae pharetra eros, quis dignissim magna. Sed tellus leo, semper non vestibulum vel, tincidunt eu mi. Aenean pretium ut velit sed facilisis. Ut placerat urna facilisis dolor suscipit vehicula. Ut ut auctor nunc. Nulla non massa eros. Proin rhoncus arcu odio, eu lobortis metus sollicitudin eu. Duis maximus ex dui, id bibendum diam dignissim id. Aliquam quis lorem lorem. Phasellus sagittis aliquet dolor, vulputate cursus dolor convallis vel. Suspendisse eu tellus feugiat, bibendum lectus quis, fermentum nunc. Nunc euismod condimentum magna nec bibendum. Curabitur elementum nibh eu sem cursus, eu aliquam leo rutrum. Sed bibendum augue sit amet pharetra ullamcorper. Aenean congue sit amet tortor vitae feugiat.
	
	Mauris interdum porttitor fringilla. Proin tincidunt sodales leo at ornare. Donec tempus magna non mauris gravida luctus. Cras vitae arcu vitae mauris eleifend scelerisque. Nam sem sapien, vulputate nec felis eu, blandit convallis risus. Pellentesque sollicitudin venenatis tincidunt. In et ipsum libero. Nullam tempor ligula a massa convallis pellentesque.
	
	\subsection{Data Analysis}
	
	Vestibulum sodales orci a nisi interdum tristique. In dictum vehicula dui, eget bibendum purus elementum eu. Pellentesque lobortis mattis mauris, non feugiat dolor vulputate a. Cras porttitor dapibus lacus at pulvinar. Praesent eu nunc et libero porttitor malesuada tempus quis massa. Aenean cursus ipsum a velit ultricies sagittis. Sed non leo ullamcorper, suscipit massa ut, pulvinar erat. Aliquam erat volutpat. Nulla non lacus vitae mi placerat tincidunt et ac diam. Aliquam tincidunt augue sem, ut vestibulum est volutpat eget. Suspendisse potenti. Integer condimentum, risus nec maximus elementum, lacus purus porta arcu, at ultrices diam nisl eget urna. Curabitur sollicitudin diam quis sollicitudin varius. Ut porta erat ornare laoreet euismod. In tincidunt purus dui, nec egestas dui convallis non. In vestibulum ipsum in dictum scelerisque.
	
	Mauris interdum porttitor fringilla. Proin tincidunt sodales leo at ornare. Donec tempus magna non mauris gravida luctus. Cras vitae arcu vitae mauris eleifend scelerisque. Nam sem sapien, vulputate nec felis eu, blandit convallis risus. Pellentesque sollicitudin venenatis tincidunt. In et ipsum libero. Nullam tempor ligula a massa convallis pellentesque. Mauris interdum porttitor fringilla. Proin tincidunt sodales leo at ornare. Donec tempus magna non mauris gravida luctus. Cras vitae arcu vitae mauris eleifend scelerisque. Nam sem sapien, vulputate nec felis eu, blandit convallis risus. Pellentesque sollicitudin venenatis tincidunt. In et ipsum libero. Nullam tempor ligula a massa convallis pellentesque.
	
	%------------------------------------------------
		
		
	\section{Introduction}
	What are knowledge graphs? Why are they important? What is the problem the authors are trying to solve by using KG in an attention network for a recommender algorithm?

	\section{Task 1}
	W pierwszym etapie projektu studenci wybierają artykuł naukowy
	z zadanej przez prowadzącą puli dostępnych artykułów i analizują go.
	W pierwszej kolejności studenci powinni skupić się na poruszanym w artykule
	problemie naukowym oraz na opisie istniejących sposobów podejścia do
	danego problemu (o ile istnieją). Jeżeli takie rozwiązania istnieją, należy
	dowiedzieć się, jakie są ich zalety oraz wady, które zachęciły autorów
	wybranego artykułu do podjęcia danej tematyki.
	Kolejnym krokiem w Etapie I jest zrozumienie zaproponowanego
	w artykule algorytmu lub metody. Studenci powinni szczegółowo przeanalizować
	przebieg algorytmu, wszystkie parametry stałe i zmienne oraz format
	przyjmowanych i zwracanych danych oraz etap procesu rekomendacji, w którym
	algorytm/metoda zostały dodane/zmodyfikowane (w zależności od wybranego
	przez studenta artykułu).
	Na zakończenie pierwszego etapu studenci powinni dokonać refleksji nad
	poznanym algorytmem oraz problemem, który miał rozwiązywać i odpowiedzieć
	na kilka pytań:
	\begin{itemize}
		\item Jakie są wady i zalety proponowanej w artykule metody?
		\item Czym różni się od wcześniejszych rozwiązań?
		\item Czy jest uniwersalna? Jeśli nie, to jakie są jej ograniczenia?
		\item Czy rzeczywiście rozwiązuje opisywany problem? Jeśli nie, to dlaczego?
		\item Czy widzisz sposób na ulepszenie/modyfikację proponowanego w artykule algorytmu? Jeśli tak, to na czym ta modyfikacja polega?
	\end{itemize}

	\subsection{Sprawozdanie do Etapu I}
	Sprawozdanie z Etapu I powinno zawierać podstawowe informacje
	o autorach, tj. imiona, nazwiska i numery indeksów wszystkich członków
	zespołu projektowego oraz informacje o analizowanym artykule, tj. nazwiska
	autorów i tytuł. 
	Poniżej podano wymagane elementy merytoryczne sprawozdania.
	\begin{enumerate}
		\item Motywacje
		\begin{enumerate}
			\item Poruszany w artykule problem z opisem, dlaczego jest ważny dla dziedziny.
			\item Istniejące rozwiązania problemu wraz z ich zaletami i wadami.
			\item Informacja, dlaczego autorzy analizowanego artykułu zajęli się tą tematyką.
			\item Dlaczego autorzy uważają, że ich metoda rozwiąże lub pomoże rozwiązać w przyszłości dany problem.
		\end{enumerate}
		\item Algorytm
		\begin{enumerate}
			\item Szczegółowy opis algorytmu, najlepiej dodatkowo zilustrowany pseudokodem lub schematem blokowym.
			\item Identyfikacja wszystkich parametrów algorytmu:
			\begin{enumerate}
				\item stałe wraz z ich wartościami
				\item zmienne wraz z możliwym zakresem (i co się zmienia dla różnych wartości)
			\end{enumerate}
			\item Identyfikacja formatu danych:
			\begin{enumerate}
				\item Wymagany format danych wejściowych.
				\item Format zwracanych danych.
			\end{enumerate}
			\item Analiza procesu rekomendacji – w którym etapie procesu rekomendacji proponowany algorytm jest wykorzystywany.
		\end{enumerate}
		\item Wnioski
		\begin{enumerate}
			\item Jakie są wady i zalety proponowanej w artykule metody?
			\item Czym różni się od wcześniejszych rozwiązań?
			\item Czy jest uniwersalna? Jeśli nie, to jakie są jej ograniczenia?
			\item Czy rzeczywiście rozwiązuje opisywany problem? Jeśli nie, to dlaczego?
			\item Czy widzisz sposób na ulepszenie/modyfikację proponowanego w artykule algorytmu? Jeśli tak, to na czym ta modyfikacja polega?
		\end{enumerate}
	\end{enumerate}

	\section{Task 2}
	W drugim etapie projektu studenci analizują kod dołączony do wybranego
	w pierwszym etapie artykułu naukowego. W pierwszej kolejności studenci
	powinni skupić się na sposobie zaimplementowania proponowanej przez
	autorów artykułu metody, w tym na poprawność implementacji oraz ciekawe
	rozwiązania techniczne.
	W kolejnym kroku studenci powinni przeanalizować sposób
	przeprowadzenia eksperymentu. Jaki zbiór danych został użyty? W jaki sposób
	dokonano podziału danych? Z jakich metod i miar skorzystano? Czy protokół
	oceny algorytmu odpowiada problemowi, który autorzy artykułu chcieli
	rozwiązać?
	W ostatniej części etapu studenci powinni powtórzyć eksperyment opisany
	w wybranym artykule z wykorzystaniem analizowanego we wcześniejszych
	krokach kodu. Co należy zrobić, aby uruchomić kod? Czy wystarczy jedno
	uruchomienie, czy np. istnieje plik konfiguracyjny, który należy modyfikować,
	aby otrzymać wszystkie raportowane w artykule wyniki? Czy wyniki zgadzają się
	z tym, które zostały zaraportowane w artykule przez autorów? Jeżeli nie, to
	dlaczego tak jest? Jakie mogą być tego konsekwencje?

	\subsection{Prezentacja do Etapu II}
	Prezentacja z Etapu II powinna zawierać podstawowe informacje
	o autorach, tj. imiona, nazwiska i numery indeksów wszystkich członków
	zespołu projektowego oraz informacje o analizowanym artykule, tj. nazwiska
	autorów i tytuł. 
	Poniżej podano elementy merytoryczne prezentacji. (Jeżeli czegoś z poniższych
	nie można zastosować dla konkretnego tematu, to powinno zostać pominięte – 
	ewentualnie umieszczone na slajdzie, ale nie omówione podczas prezentacji.)
	\begin{enumerate}
		\item Omówienie algorytmu, czyli streszczenie etapu 1 celem zrozumienia
		dalszej części prezentacji przez kolegów (tylko i aż tyle!).
		\item Analiza kodu
		\begin{enumerate}
			\item Poprawność implementacji (błędy metody i merytoryczne, tj.
			niezgodność z opisem metody).
			\item Ciekawe rozwiązania techniczne, np. optymalizacja obliczeń itp.
		\end{enumerate}
		\item Opis użytego w eksperymencie zbioru/zbiorów danych
		\begin{enumerate}
			\item Jaki był cel zbierania zbioru?
			\item Kto jest twórcą/właścicielem zbioru?
			\item Jakie dane zawiera?
			\item Co one oznaczają?
			\item Jakie są wartości dla podstawowych charakterystyk zbioru danych?
			\begin{enumerate}
				\item Ilość ocen, użytkowników i produktów.
				\item Średnie ilości ocen na użytkownika/produkt.
				\item Gęstość macierzy ocen.
				\item Dodatkowe informacje zawarte w zbiorze charakterystyczne dla
				rozwiązywanego problemu/proponowanej metody.
			\end{enumerate}
		\end{enumerate}
		\item Analiza eksperymentu
		\begin{enumerate}
			\item Z jakich metod ewaluacji skorzystano?
			\item Z jakich miar oceny systemów rekomendacyjnych skorzystano? Dlaczego?
			\item Czy wykorzystano odpowiedni zbiór danych do rozwiązywanego
			problemu?
			\item W jaki sposób dokonano podziału danych? Czy przeprowadzano na
			nich jakieś dodatkowe transformacje?
			\item Czy protokół oceny algorytmu odpowiada problemowi, który autorzy
			artykułu chcieli rozwiązać?
		\end{enumerate}
		\item Wykonanie kodu
		\begin{enumerate}
			\item W jaki sposób uruchamia się program?
			\item Czy istnieją jakieś pliki konfiguracyjne? Jakie? Do czego służą?
			\item Jak długo trzeba czekać na wyniki? (nie wymaga się podawania
			dokładnych danych).
		\end{enumerate}
		\item Wyniki i wnioski
		\begin{enumerate}
			\item Wyniki własnego wykonania w zestawieniu z wynikami z artykułu.
			\item Czy wyniki się różnią?
			\item Jaka może być tego przyczyna?
			\item Jakie mogą być tego konsekwencje?
			\item Czy uważasz, że algorytm został poprawnie zwalidowany? Dlaczego?
			\item Czy widzisz sposób na ulepszenie/modyfikację tego eksperymentu?
		\end{enumerate}
	\end{enumerate}

	\section{Task 3}
	W trzecim etapie projektu studenci, korzystając z wiedzy zdobytej 
	w poprzednich dwóch etapach projektu, proponują własną modyfikację algorytmu 
	i/lub eksperymentu przeprowadzonego przez autorów oryginalnego artykułu i 
	planują eksperyment mający na celu weryfikację zaproponowanej modyfikacji.
	Na początku studenci powinni przemyśleć możliwe modyfikacje, które 
	zidentyfikowali w poprzednich sprawozdaniach. Można też rozszerzyć pulę 
	o nowe propozycje. Studenci powinni opisać wszystkie zidentyfikowane przez 
	siebie możliwe modyfikacje, zarówno samego algorytmu, jak i eksperymentu (nie 
	obliguje to do implementacji wszystkich rozszerzeń w kolejnym etapie).
	Podczas opisu każdej z możliwych modyfikacji algorytmu i/lub 
	eksperymentu, studenci powinni rozważyć następujące kwestie:
	\begin{itemize}
		\item Na czym polega dana modyfikacja?
		\item W jaki sposób zmieni ona sposób wykonania algorytmu/eksperymentu?
		\item W którym miejscu w kodzie będą wprowadzone zmiany?
		\item Jakie są motywacje wprowadzenia danej modyfikacji?
		\item Czy pozwoli ona ulepszyć algorytm/eksperyment? W jaki sposób?
		\item W przypadku modyfikacji eksperymentu powinno się rozważyć również pytania:
		\begin{itemize}
			\item Jaką nową wiedzę o metodzie możemy zyskać dzięki danej modyfikacji eksperymentu?
			\item Do czego w praktyce może się ta wiedza przydać?
		\end{itemize}
	\end{itemize}
	W ostatniej części etapu trzeciego studenci wybierają dwie modyfikacje i 
	planują dla nich eksperymenty. Należy również uzasadnić wybór modyfikacji do 
	implementacji.
	Plan eksperymentu powinien zawierać informacje o tym, co będzie badane 
	i w jaki sposób. Studenci powinni zidentyfikować wszystkie zmienne zależne i
	niezależne, w tym również hiperparametry wymagające wcześniejszego 
	strojenia. Każdy krok eksperymentu powinien być dokładnie zaplanowany i 
	opisany. Ważny jest również wybór i opis zbiorów danych do eksperymentu.

	\subsection{Prezentacja do Etapu III}
	Prezentacja z Etapu III powinna zawierać podstawowe informacje 
	o autorach, tj. imiona, nazwiska i numery indeksów wszystkich członków zespołu 
	projektowego oraz informacje o analizowanym artykule, tj. nazwiska autorów i 
	tytuł. 
	Poniżej podano wymagane elementy merytoryczne sprawozdania.
	Prezentacja powinna trwać 10-12 minut.
	\begin{enumerate}
		\item Możliwe modyfikacje
		\begin{itemize}
			\item Dla każdej zidentyfikowanej możliwej modyfikacji, studenci powinni rozważyć następujące elementy:
			\begin{enumerate}
				\item Istota modyfikacji
				\begin{itemize}
					\item Typ modyfikacji (algorytmu czy eksperymentu).
					\item Na czym polega dana modyfikacja?
					\item W jaki sposób zmieni ona sposób wykonania algorytmu/eksperymentu?
					\item W którym miejscu w kodzie będą wprowadzone zmiany?
				\end{itemize}
				\item Motywacje do wprowadzenia danej modyfikacji
				\begin{itemize}
					\item Jakie są motywacje wprowadzenia danej modyfikacji?
					\item Czy pozwoli ona ulepszyć algorytm/eksperyment? W jaki sposób?
					\item Opcjonalne (tylko w przypadku modyfikacji eksperymentu): Jaką nową wiedzę o metodzie możemy zyskać dzięki danej modyfikacji eksperymentu? Do czego w praktyce może się ta wiedza przydać?
				\end{itemize}
			\end{enumerate}
		\end{itemize}
		\item Plan eksperymentu
		\begin{enumerate}
			\item Wybór modyfikacji
			\begin{itemize}
				\item Która z powyższych modyfikacji została wybrana do implementacji? Dlaczego akurat ta?
			\end{itemize}
			\item Opis eksperymentu
			\begin{itemize}
				\item Dokładny opis planowanego eksperymentu.
				\item Jakie zbiory danych zostaną wykorzystane?
				\item Jakie kroki zostaną przeprowadzone?
				\item Co będziemy badać?
				\item Jakie będą zmienne zależne a jakie niezależne w projektowanym eksperymencie?
				\item Czy będą wykorzystywane miary jakości systemów rekomendacyjnych? Jeśli tak, to jakie?
				\item Czy wymagane jest wcześniejsze strojenie hiperparametrów modelu? Jeżeli tak, to których i w jaki sposób będzie wykonane?
			\end{itemize}
		\end{enumerate}
	\end{enumerate}

	\section{Task 4}
	W czwartym etapie projektu studenci, korzystając z planu eksperymentu
	zaprojektowanego w trzecim etapie, przeprowadzają eksperyment naukowy,
	analizują jego wyniki i formułują wnioski.
	W pierwszej części czwartego etapu studenci powinni wykonać dwa
	eksperymenty zgodnie z planem opisanym w sprawozdaniu do etapu trzeciego.
	Istotna jest tutaj zgodność z opisanym planem. Jeżeli wystąpią jakiekolwiek
	trudności z realizacją zaplanowanych w eksperymencie czynności, studenci
	powinni opisać te problemy wraz z rozwiązaniem, które zostało zastosowane.
	Wszelkie modyfikacje planu również powinny zostać skrupulatnie opisane.
	Po wykonaniu każdego z eksperymentów, studenci powinni zebrać
	otrzymane wyniki w wygodnej do analizy formie, np. wykres, tabela. Dobrze by
	było, gdyby zostały przedstawione w takiej formie, która umożliwia porównanie
	otrzymanych wyników z tymi z oryginalnego artykułu (w uzasadnionych
	przypadkach nie jest to konieczne). Otrzymane wyniki powinny zostać
	przeanalizowane samodzielnie, jak również porównane z tymi z oryginalnego
	artykułu (o ile to możliwe). Na podstawie planu modyfikacji i wyników oraz
	oryginalnego artykułu, studenci powinni wyciągnąć wnioski z przeprowadzonych
	badań (dla każdego eksperymentu osobno).
	W ostatniej części etapu czwartego studenci podsumowują pracę ze
	wszystkich dotychczasowych etapów i wyciągają wnioski końcowe. Kluczowe
	jest zidentyfikowanie, czy podczas przeprowadzonych eksperymentów powstała
	nowa wiedza o badanym algorytmie (i opisanie jej, jeśli tak). Możliwe jest tu
	uwzględnienie wszelkich uwag dotyczących oryginalnej metody, procedury
	ewaluacji, zaimplementowanych modyfikacjach i otrzymanych wyników.

	\subsection{Prezentacja do Etapu IV}
	Prezentacja z Etapu IV powinna zawierać podstawowe informacje
	o autorach, tj. imiona, nazwiska i numery indeksów wszystkich członków
	zespołu projektowego oraz informacje o analizowanym artykule, tj. nazwiska
	autorów i tytuł. 
	Poniżej podano ramowe elementy merytoryczne sprawozdania.
	\begin{enumerate}
		\item Analizowany algorytm
		\begin{itemize}
			\item Krótkie przypomnienie, co robi analizowany algorytm.
		\end{itemize}
		\item Eksperymenty
		\begin{enumerate}
			\item Przebieg eksperymentu
			\begin{itemize}
				\item Dokładny opis przeprowadzonego eksperymentu.
				\item Czy udało się przeprowadzić modyfikację zgodnie z założonym planem? Jeśli nie, to co się zmieniło?
				\item Czy wystąpiły jakieś niespodziewane problemy? Jeśli tak, to jakie?
			\end{itemize}
			\item Wyniki
			\begin{itemize}
				\item Dokładny opis otrzymanych wyników z eksperymentu wraz z tabelami i/lub wykresami.
			\end{itemize}
			\item Porównanie wyników i wnioski
			\begin{itemize}
				\item Czy wyniki różnią się od wyników otrzymanych przez autorów oryginalnego artykułu? Jeśli tak, to czy są lepsze czy gorsze? Jak myślisz, dlaczego tak jest?
				\item Jeśli modyfikacją było nowe zastosowanie metody, to czy po przeprowadzonym eksperymencie możemy uznać, że metoda sprawdzi się w tym zastosowaniu? Dlaczego? Itd.
			\end{itemize}
		\end{enumerate}
		\item Wnioski końcowe
		\begin{itemize}
			\item Co możesz powiedzieć o oryginalnym algorytmie i własnych modyfikacjach po przeprowadzeniu eksperymentów?
			\item Czy możesz wysnuć jakieś wnioski z wszystkich eksperymentów (z artykułu i własnych)?
			\item Czy powstała nowa wiedza podczas realizacji projektu? Jeśli tak, to jaka? Itp.
		\end{itemize}
	\end{enumerate}

	\section{Task 5}
	\subsection{Przebieg Etapu V}
	W piątym etapie projektu studenci przygotowują artykuł naukowy zgodnie
	z załączonym szablonem.
	Artykuł powinien zawierać wszystkie kluczowe informacje dotyczące
	wszystkich wcześniejszych etapów projektu, ze szczególnym naciskiem na etap
	czwarty (własny eksperyment i wnioski).

	\subsection{Sprawozdanie do Etapu V}
	Sprawozdanie z Etapu V będzie stanowił artykuł naukowy, który powinien
	zawierać podstawowe informacje o autorach, tj. imiona i nazwiska wszystkich
	członków zespołu projektowego oraz informacje o analizowanym artykule, tj.
	nazwiska autorów i tytuł. Ponadto, artykuł powinien zawierać następujące
	elementy:
	\begin{enumerate}
		\item Wprowadzenie do problematyki.
		\item Przegląd stanu wiedzy/istniejących modyfikacji algorytmu
		\item Analizowany algorytm
		\item Przeprowadzone eksperymenty wraz z wynikami
		\item Wnioski końcowe
	\end{enumerate}
	Dokładny dobór treści w ramach punktów jest pozostawiony studentom
	i podlega ocenie.
	Artykuł powinien być napisany w języku angielskim z wykorzystaniem
	załączonego szablonu LaTeX. Długość artykułu powinna wynosić 10-14 stron
	(łącznie z bibliografią).

	\subsection{Forma i sposób zaliczenia Etapu V}
	Za etap można otrzymać maksymalnie 8 punktów. Do zaliczenia zadania
	wymagane jest otrzymanie minimum 4 punkty. Oceniany będzie dobór i sposób
	zaprezentowania treści. Termin wgrania artykułów upływa 21.01.2024 o godz.
	23:59.

	%----------------------------------------------------------------------------------------
	%	 REFERENCES
	%----------------------------------------------------------------------------------------
	
	\printbibliography % Output the bibliography
	
	%----------------------------------------------------------------------------------------
	
\end{document}